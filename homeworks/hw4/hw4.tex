\documentclass[10pt]{article}

\usepackage{graphicx}
\usepackage{amsmath,amsfonts,amssymb}

% use different colors for links:
\usepackage{color}
\definecolor{darkgreen}{rgb}{0.1,0.5,0.1}
\definecolor{darkblue}{rgb}{0.2,0.2,1.0}
\usepackage[colorlinks=true,linkcolor=darkblue,citecolor=darkblue,
            filecolor=darkblue,urlcolor=darkgreen]{hyperref}


\setlength{\textwidth}{6.2in}
\setlength{\oddsidemargin}{0.3in}
\setlength{\evensidemargin}{0in}
\setlength{\textheight}{8.9in}
\setlength{\voffset}{-1in}
\setlength{\headsep}{26pt}
\setlength{\parindent}{0pt}
\setlength{\parskip}{5pt}



\input{../latex/macros.tex}  % input some useful macros

\begin{document}

% header:
\hfill\vbox{\hbox{AMath 586 / ATM 581}
\hbox{Homework \#3}\hbox{Due Tuesday, May 12, 2015}}

{\bf Name:} Your name here
\vskip 5pt

Homework is due to Canvas by 11:00pm PDT on the due date.

To submit, see \url{https://canvas.uw.edu/courses/962872/assignments/2860108}


%--------------------------------------------------------------------------
\vskip 1cm
\hrule
{\bf Problem 1}  


Consider the following method for solving the heat equation
$u_t=u_{xx}$:
\[
U_i^{n+2} = U_i^n + \frac{2k}{h^2}(U_{i-1}^{n+1} - 2U_i^{n+1} +
U_{i+1}^{n+1}).
\]
\begin{enumerate}
\item Determine the formal order of accuracy of this method 
(in both space and time) based on computing the local truncation error.

\item Suppose we take $k=\alpha h^2$ for some fixed $\alpha>0$ and refine
the grid.  Show that this method fails to be 
Lax-Richtmyer stable for any choice of $\alpha$.

Do this in two ways: 
\begin{itemize}
\item Consider the MOL interpretation and the stability region of
the time-discretization being used.
\item Use von Neumann analysis and solve a quadratic equation for $g(\xi)$.
\end{itemize} 

\item What if we take $k=\alpha h^3$ for some fixed $\alpha>0$ and refine
the grid. Would this method be stable?

\end{enumerate}


% uncomment the next two lines if you want to insert solution...
%\vskip 1cm
%{\bf Solution:}

% insert your solution here!

%--------------------------------------------------------------------------
\vskip 1cm
\hrule
{\bf Problem 2}  


Consider the PDE
\begin{equation}\label{diffdecay}
u_t = \kappa u_{xx} - \gamma u,
\end{equation}
which models diffusion combined with decay provided $\kappa>0$ and $\gamma>0$.  
Consider methods of the form
\begin{equation}\label{ddtheta}
U_j^{n+1} =U_j^n+{k\over 2h^2} [U_{j-1}^n-2U_j^n +U_{j+1}^n
+U_{j-1}^{n+1} - 2U_j^{n+1} + U_{j+1}^{n+1}]
- k\gamma[(1-\theta)U_j^n + \theta U_j^{n+1}]
\end{equation}
where $\theta$ is a parameter.  In particular, if $\theta=1/2$ then the
decay term is modeled with the same centered-in-time approach as the
diffusion term and the method can be obtained by applying the Trapezoidal
method to the MOL formulation of the PDE.   If $\theta=0$ then the decay
term is handled explicitly.  For more general reaction-diffusion equations
it may be advantageous to handle the reaction terms explicitly since these
terms are generally nonlinear, so making them implicit would require solving
nonlinear systems in each time step (whereas handling the diffusion term
implicitly only gives a linear system to solve in each time step).

\begin{enumerate}
\item By computing the local truncation error, show that this method is
$\bigo(k^p+h^2)$ accurate, where $p=2$ if $\theta = 1/2$ and $p=1$
otherwise.
\item Using von Neumann analysis, show that this method is unconditionally
stable if $\theta \geq 1/2$.
\item Show that if $\theta = 0$ then the method is stable provided $k\leq
2/\gamma$, independent of $h$.
\end{enumerate} 


% uncomment the next two lines if you want to insert solution...
%\vskip 1cm
%{\bf Solution:}

% insert your solution here!

%--------------------------------------------------------------------------

\vskip 1cm
\hrule
{\bf Problem 3}  


\begin{enumerate} 
\item The m-file \verb+heat_CN.m+ from the book repository
\url{http://faculty.washington.edu/rjl/fdmbook/} 
solves the heat equation $u_t = \kappa u_{xx}$  (with $\kappa = 0.02$)
using the Crank-Nicolson method.
A Python version is available in \verb+$AM586/codes/heat_CN.py+.

Run this code, and by changing the number of grid points, confirm that it is
second-order accurate.  (Observe how the error at some fixed time such as $T=1$
behaves as $k$ and $h$ go to zero with a fixed relation between $k$ and $h$,
such as $k = 4h$.)

Note that in order for the time step to evenly define the time interval the
way this code is set up, you need to specify values such as $m=19, 39, 79$.
(The number of interior grid points.)

Produce a log-log plot of the error versus $h$.

\item Modify this code to produce a new version that
implements the TR-BDF2 method on the same problem.  Test it to confirm that
it is also second order accurate.  Explain how you determined the proper
boundary conditions in each stage of this Runge-Kutta method.

\item Modify \verb+heat_CN.m+ to produce a new m-file \verb+heat_FE.m+ that
implements the forward Euler explicit 
method on the same problem.  Test it to confirm that
it is $\bigo(h^2)$ accurate as $h\goto 0$ provided when $k = 24 h^2$ is
used, which is within the stability limit for $\kappa = 0.02$.  Note how
many more time steps are required than with Crank-Nicolson or TR-BDF2,
especially on finer grids.

\item Test \verb+heat_FE.m+ with $k = 26 h^2$, for which it should be
unstable.  Note that the instability does not become apparent until about
time 4.5 for the parameter values $\kappa = 0.02,~ m=39,~\beta = 150$.
Explain why the instability takes several hundred time steps to appear, and
why it appears as a sawtooth oscillation. 

{\bf Hint:} What wave numbers $\xi$ are growing exponentially for these
parameter values?  What is the initial magnitude of the most unstable
eigenmode in the given initial data?  The expression (E.30) for the Fourier
transform of a Gaussian may be useful.

\end{enumerate}



% uncomment the next two lines if you want to insert solution...
%\vskip 1cm
%{\bf Solution:}

% insert your solution here!

%--------------------------------------------------------------------------

\vskip 1cm
\hrule
{\bf Problem 4}  


\begin{enumerate}

\item Modify \verb+heat_CN.m+  (or \verb+heat_CN.py+)
to solve the heat equation for
$-1\leq x \leq 1$ with step function  initial data
\begin{equation} \label{9.3a}
u(x,0) = \begin{choices}  1 \when x<0\\  0 \when x\geq 0. \end{choices}
\end{equation} 
With appropriate Dirichlet boundary conditions, the exact solution is
\begin{equation} \label{9.3b}
u(x,t) = \half \, \text{erfc} \left(x / \sqrt{4 \kappa t}\right)
\end{equation} 
for $t>0$, where erfc is the {\em complementary error function}
\[
\text{erfc}(x) = \frac{2}{\sqrt{\pi}} \int_x^\infty e^{-z^2}\,dz.
\]
Note that {\tt erfc} is a built-in Matlab function and in Python you could
use \verb+scipy.special.erfc+.

\begin{enumerate}
\item
Use $\kappa = 0.02$ and
test this method using $m=39$ and $k = 10h$ for $0\leq x \leq 1$.  
Note that there is an initial rapid transient decay of the high
wave numbers that is not captured well with this size time step.

\item
How small do you need to take the time step to get reasonable results?
For a suitably small time step, explain why you get much better results by
using $m=38$ than $m=39$.  What is the observed order of accuracy as $k\goto
0$ when $k = \alpha h$ with $\alpha$ suitably small and $m$ even?

Hint: You might also see what you get if you redefine the initial data so
that $u(0,0)=1/2$ with $u(x,0)$ the same everywhere else, rather than
$u(0,0) = 0$.

\end{enumerate}

\item Modify your TR-BDF2 solver from Problem 3
to solve the heat equation for
$-1\leq x \leq 1$ with step function  initial data as above.
Test this routine using $k=10h$ and estimate the order of accuracy as
$k\goto 0$ with $m$ even.  Why does the TR-BDF2 method work better than
Crank-Nicolson?

\end{enumerate} 


% uncomment the next two lines if you want to insert solution...
%\vskip 1cm
%{\bf Solution:}

% insert your solution here!


%--------------------------------------------------------------------------

\end{document}

