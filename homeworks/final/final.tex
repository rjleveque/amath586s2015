\documentclass[10pt]{article}

\usepackage{graphicx}
\usepackage{amsmath,amsfonts,amssymb}

% use different colors for links:
\usepackage{color}
\definecolor{darkgreen}{rgb}{0.1,0.5,0.1}
\definecolor{darkblue}{rgb}{0.2,0.2,1.0}
\usepackage[colorlinks=true,linkcolor=darkblue,citecolor=darkblue,
            filecolor=darkblue,urlcolor=darkgreen]{hyperref}


\setlength{\textwidth}{6.2in}
\setlength{\oddsidemargin}{0.3in}
\setlength{\evensidemargin}{0in}
\setlength{\textheight}{8.9in}
\setlength{\voffset}{-1in}
\setlength{\headsep}{26pt}
\setlength{\parindent}{0pt}
\setlength{\parskip}{5pt}



\input{../latex/macros.tex}  % input some useful macros

\begin{document}

% header:
\hfill\vbox{\hbox{AMath 586 / ATM 581}
\hbox{Take-home Final}\hbox{Due Thursday, June 11, 2015}}

{\bf Name:} Your name here
\vskip 5pt

This exam is due to Canvas by 11:00pm PDT on the due date.

To submit, see \url{https://canvas.uw.edu/courses/962872/assignments/2878942}

{\bf Please work on this exam on your own!}  You may consult the literature
and use the discussion board and office hours if questions arise.


%--------------------------------------------------------------------------
\vskip 1cm
\hrule
{\bf Problem 1}  

Consider the method
\begin{equation} \label{a}
U_j^{n+1} = U_j^n - \frac{ak}{2h}(U_j^n-U_{j-1}^n + U_j^{n+1}-U_{j-1}^{n+1}).
\end{equation}
for the advection equation $u_t+au_x=0$ on $0\leq x \leq 1$ with periodic
boundary conditions, on a grid with spacing $h= 1 / (m+1)$ for some integer
$m$.  

\begin{enumerate}
\item This method can be viewed as the trapezoidal method applied to an ODE
system $U'(t) = AU(t)$ arising from a method of lines discretization of the
advection equation.  What is the matrix $A$?  Don't forget the boundary
conditions.

\item Determine the eigenvalues of the matrix $A$.  {\bf Hint:} the
eigenvectors are the same as for similar matrices you have seen.

\item Suppose we want to fix the Courant number $ak/h$ as $k,~h\goto 0$.
For what range of Courant numbers will the method be stable if $a>0$?
If $a<0$?  Justify your answers in terms of eigenvalues of the matrix $A$
and  the stability region of the trapezoidal method.

\item For what range of $ak/h$ will the CFL condition be satisfied for this
method (with periodic boundary conditions)?  Consider both $a>0$ and $a<0$.
Is satisfying the CFL condition sufficient for stability in all cases?

\end{enumerate} 



% uncomment the next two lines if you want to insert solution...
%\vskip 1cm
%{\bf Solution:}

% insert your solution here!



%--------------------------------------------------------------------------
\vskip 1cm
\hrule
{\bf Problem 2}  

Consider the dispersive PDE $u_t + \gamma u_{xxx} = 0$ with periodic boundary
conditions on an interval $a \leq x \leq b$.  

\begin{enumerate} 
\item The third-derivative term could be approximated by either 
\begin{equation} \label{2a}
u_{xxx}(x_j,t_n) \approx \frac 1 {h^3}(-U_{j-2}^n + 3U_{j-1}^n - 3U_j^n + U_{j+1}^n)
\end{equation} 
or by 
\begin{equation} \label{2b}
u_{xxx}(x_j,t_n) \approx \frac 1 {h^3}(-U_{j-1}^n + 3U_{j}^n - 3U_{j+1}^n + U_{j+2}^n)
\end{equation} 
Show that either one is first-order accurate.  

\item Suppose you plan to use a trapezoid discretization in time for a
problem where $\gamma >0$.  Which form of spatial discretization
\eqn{2a} or \eqn{2b} would be better?  Justify your answer.

\item Implement a Crank-Nicolson type method for this equation, which uses
the trapezoid method in time together with the spatial discretization from
above that you determined is best for the case $\gamma >0$.
Your implementation should allow arbitrary values of the endpoints $a$ and
$b$ of the spatial interval, since two different choices are required below.

Test your method by showing it is first order accurate 
for initial data $u(x,0) = \sin(px)$ for integer $p$ such as $p=2$ on the
interval $0 \leq x \leq 2\pi$.  Note that 
you can compute the exact solution for such data.
\end{enumerate} 


% uncomment the next two lines if you want to insert solution...
%\vskip 1cm
%{\bf Solution:}

% insert your solution here!



%--------------------------------------------------------------------------
\vskip 1cm
\hrule
{\bf Problem 3}  

Now consider the KdV equation $u_t + 6uu_x + u_{xxx} = 0$.  
This Korteweg - de Vries equation is famous --- read about it if you are not
familiar with it.  Even though this equation is nonlinear,
there are exact ``solitary wave'' solutions that consist of a translating wave of
the form $u(x,t) = \eta(x-\alpha^2 t)$ for special initial data of the form
\begin{equation} \label{3a}
\eta(x) = \frac {\alpha^2}{2} \text{sech}^2\left(\alpha x/2\right)
\end{equation} 
for any choice of parameter $\alpha >0$.  Note that taller pulses
propagate more rapidly.  An amazing feature of such problems is the
way two such pulses interact as ``solitons''. 

Construct a first-order accurate method by combining your solution to Problem 2
with a first-order upwind method for the hyperbolic equation
\begin{equation} \label{3b}
u_t + 3(u^2)_x = 0.
\end{equation} 
Use the first-order fractional step approach to combine the methods --- i.e.
in each time step, first advance the solution by taking a step with the $u_t
+ u_{xxx} = 0$ to get an intermediate solution $U^*$, and then use this as
initial data for a step in which you solve \eqn{3b} by the upwind method for
this nonlinear conservation law.  

Again use periodic boundary conditions.  The true solution is not periodic,
but decays exponentially to zero so as long as the interval is long enough
relative to the width of the solitary wave, so this should not affect errors
computed using a periodic extension of \eqn{3a}.

Test your method on the interval $-1 \leq x \leq 3$ with $\alpha=10$ out to
time $t=0.02$.  Demonstrate that your method is stable and first-order
accurate by refining the grid with $r = k/h$ fixed.

Estimate the maximum stable value of $r$ based on stability theory,
justifying your estimate, and confirm that this is consistent with
what you observe from your numerical method.

\vskip 10pt
Note that this is not a very accurate method!  If you want to play around
with soliton interaction (optional) you might try the following data:
\begin{equation} 
\eta(x) = \frac {\alpha_1^2}{2} \text{sech}^2\left(\alpha_1 (x + 2)/2
\right) + \frac {\alpha_2^2}{2} \text{sech}^2\left(\alpha_2 (x + 1)/2\right)
\end{equation} 
with $\alpha_1 = 25, ~\alpha_2= 16$ on the interval $-3\leq x \leq 2$.  

If you want to read about (and try out) a much more accurate method
for this problem, see for example Program 27 in Chapter 10 of
``Spectral Methods in Matlab'' by L.\ N.\ Trefethen (SIAM 2000),
available as an ebook at
\url{http://epubs.siam.org/doi/book/10.1137/1.9780898719598}.
The programs are available at
\url{http://people.maths.ox.ac.uk/trefethen/spectral.html}.



% uncomment the next two lines if you want to insert solution...
%\vskip 1cm
%{\bf Solution:}

% insert your solution here!


\end{document}

