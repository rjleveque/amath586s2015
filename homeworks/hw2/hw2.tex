\documentclass[10pt]{article}

\usepackage{graphicx}
\usepackage{amsmath,amsfonts,amssymb}

\usepackage{hyperref}  % for urls and hyperlinks


\setlength{\textwidth}{6.2in}
\setlength{\oddsidemargin}{0.3in}
\setlength{\evensidemargin}{0in}
\setlength{\textheight}{8.9in}
\setlength{\voffset}{-1in}
\setlength{\headsep}{26pt}
\setlength{\parindent}{0pt}
\setlength{\parskip}{5pt}



\input{../latex/macros.tex}  % input some useful macros

\begin{document}

% header:
\hfill\vbox{\hbox{AMath 586 / ATM 581}
\hbox{Homework \#2}\hbox{Due Thursday, April 16, 2015}}

\vskip 5pt

Homework is due to Canvas by 11:00pm PDT on the due date.

To submit, see \url{https://canvas.uw.edu/courses/962872/assignments/2845007}


%--------------------------------------------------------------------------
\vskip 1cm
\hrule
{\bf Problem 1}  (Corrected 11 April 2015)


The proof of convergence of 1-step methods in Section 6.3 shows that the 
global error goes to zero as $k\goto 0$.
However, this bound may be totally useless in estimating the actual error
for a practical calculation.

For example, suppose we solve $u'=-10u$ with $u(0)=1$ up to time $T=10$, so
the true solution is $u(T)=e^{-100} \approx 3.7\times 10^{-44}$.
Using forward Euler with a time step $k=0.01$,  
the computed solution is $U^N = (.9)^{1000}\approx 1.75 \times
10^{-46}$, and so $E^N \approx 3.7 \times 10^{-44}$.
Since $L=10$ for this problem, the error bound (6.16) gives
\begin{equation}\label{eq:1}
\|E^N\| \leq e^{100}\cdot 10 \cdot \|\tau \|_\infty \approx 2.7 \times
10^{44} \|\tau\|_\infty.
\end{equation} 
Here $\|\tau\|_\infty =|\tau^0| \approx 50 k$, so this upper bound on the
error does go to zero as $k\goto 0$, but obviously it is not a realistic
estimate of the error. It is too large by a factor of about $10^{88}$.

The problem is that the estimate (6.16) is based on the Lipschitz
constant $L=|\lambda|$, which gives a bound that grows exponentially in time
even when the true and computed solutions are decaying exponentially.

\begin{enumerate}
\item Determine the computed solution and error bound (6.16) for the problem
$u' = 10u$ with $u(0)=1$ up to time $T=10$.  {\em Correction: Ignore the
following statement, since it's not correct...}  Note that the error bound is
the same as in the case above, but now it is a reasonable estimate of the
actual error.

\item A more realistic error bound for the case where $\lambda<0$ can be
obtained by rewriting (6.17) as
\[
\unp = \Phi(U^n)
\]
and then determining the Lipschitz constant for the function $\Phi$.
Call this constant $M$.  Prove that if $M\leq 1$  and $E^0=0$ then
\[
|E^n| \leq  T\|\tau\|_\infty
\]
for $nk\leq T$, a bound that is similar to (6.16) but without the
exponential term.


\item Show that for forward Euler applied to $u'=\lambda
u$ we can take $M = |1+k\lambda|$.  Determine $M$ for the case $\lambda =
-10$ and $k=0.01$ and use this in the bound from part (b).
Note that this is much better than the bound \eqn{eq:1}.
{\em But note that it's still not a very sharp bound.}

\end{enumerate}





% uncomment the next two lines if you want to insert solution...
%\vskip 1cm
%{\bf Solution:}

% insert your solution here!

%--------------------------------------------------------------------------
\vskip 1cm
\hrule
{\bf Problem 2}

Which of the following Linear Multistep Methods are convergent?  For 
the ones that are not, are they inconsistent, or not zero-stable, or both?
 \begin{enumerate}
 \item $U^{n+2} = \half U^{n+1} + \half U^{n} + 2kf(U^{n+1})$
 \item $\unp = U^n$ 
 \item $U^{n+4} = U^{n} + \frac 4 3 k(f(U^{n+3})+f(U^{n+2})+f(U^{n+1}))$
 \item $U^{n+3} = -U^{n+2} + U^{n+1} +U^{n}+2k(f(U^{n+2})+f(U^{n+1}))$.
 \end{enumerate}



% uncomment the next two lines if you want to insert solution...
%\vskip 1cm
%{\bf Solution:}

% insert your solution here!


%--------------------------------------------------------------------------
\vskip 1cm
\hrule
{\bf Problem 3}

\begin{enumerate}
\item Determine the general solution to the linear difference equation
$2U^{n+3} - 5U^{n+2} + 4U^{n+1} - U^n = 0$.

{\bf Hint:} One root of the characteristic polynomial is at $\zeta=1$.

\item Determine the solution to this difference equation with the starting
values $U^0=11$, $U^1=5$, and $U^2=1$.  What is $U^{10}$?

\item Consider the LMM
\[
2U^{n+3} - 5U^{n+2} + 4U^{n+1} - U^n = k(\beta_0 f(U^n) + \beta_1 f(U^{n+1})).
\]
For what values of $\beta_0$ and $\beta_1$ is local truncation error
$\bigo(k^2)$?

\item Suppose you use the values of $\beta_0$ and $\beta_1$ just determined
in this LMM.  Is this a convergent method?

\end{enumerate} 

% uncomment the next two lines if you want to insert solution...
%\vskip 1cm
%{\bf Solution:}

% insert your solution here!


%--------------------------------------------------------------------------
\vskip 1cm
\hrule
{\bf Problem 4}

Consider the midpoint method $\unp = \unm + 2kf(U^n)$ applied to the test
problem $u' = \lambda u$.  The method is zero-stable and second order
accurate, and hence convergent.  If $\lambda<0$ then the true solution
is exponentially decaying.

On the other hand, for $\lambda<0$ and $k>0$ the point $z=k\lambda$ is never
in the region of absolute stability of this method (see Example 7.7),
and hence it seems that the numerical solution should be growing
exponentially for any nonzero time step.  (And yet it converges to
a function that is exponentially decaying.)

\begin{enumerate} 
\item
Suppose we take $U^0=\eta$, use Forward Euler to generate $U^1$, and then
use the midpoint method for $n=2,~3,~\ldots$.  Work out the exact solution
$U^n$ by solving the linear difference equation and explain how the apparent
paradox described above is resolved.

\item Devise some numerical experiments to illustrate the resolution of the
paradox.
\end{enumerate} 



% uncomment the next two lines if you want to insert solution...
%\vskip 1cm
%{\bf Solution:}

% insert your solution here!


%--------------------------------------------------------------------------
\vskip 1cm
\hrule
{\bf Problem 5}


Perform numerical experiments to confirm the claim made in Example 7.10.


% uncomment the next two lines if you want to insert solution...
%\vskip 1cm
%{\bf Solution:}

% insert your solution here!


\end{document}

